\documentclass[]{resume}

\begin{document}

% Header
\begin{center}
	\textcolor{navy}{\Huge \textbf{Ivan Kharitonov}} \par\vspace{0.2cm}
	\normalsize \textbf{ML/Research Engineer} \hfill
	ipkharitonov@gmail.com \hfill
	+995 599 008 123 \hfill
	\href{https://www.linkedin.com/in/ivan-kharitonov-main/}{LinkedIn} \hfill
	\href{https://github.com/kharitonov-ivan}{GitHub}
\end{center}
\vspace{-0.5em}

% Summary
\begin{quote}
	\noindent M.S. in Electrical Engineering with a specialization in Machine Learning. 10 years of experience in software engineering, 8 years in Machine Learning, and 4 years in Autonomous Vehicles. Led a research team for 2 years, delivering a deep learning-based motion prediction model to a production fleet. Contributing to fusion energy research using RL and generative models, with a recent publication on plasma control through deep reinforcement learning. Previous experience includes generative NLP projects for text recognition and processing. Served as a Teaching Assistant for Reinforcement Learning courses. Authorized to work in the UK under a Global Talent visa.
\end{quote}

% Experience
\section*{\sectionformat Experience}
\position{Senior research engineer}{Next Step Fusion \href{https://nextfusion.org/}{info}}{May 2023-present}
\project
{RL-based plasma shape control}
{Research and development of ML-based plasma control systems. Applied reinforcement learning to obtain policies for controlling plasma parameters. Designed simulation environments for policy training and evaluation. Collaborated with physics teams to ensure model alignment with physical principles. Developed a continuous control RL pipeline to maintain plasma shape. Conducted testbed experiments to validate simulation results.}
{Successfully tested control system on the DIII-D tokamak, demonstrating real-world applicability of ML approaches in fusion reactors. Results were published in a peer-reviewed scientific paper.}
{Python, RL, PyTorch, ray/rllib, PPO, SAC, TD3}

\project
{Multi-objective optimization of plasma states}
{Conducted research on plasma state optimization using multi-objective Bayesian optimization approaches. Explored optimization in latent space to improve search efficiency. Developed active learning methods to maximize information gain from computationally expensive simulations. Created a framework for efficient exploration of plasma parameter space.}
{}
{Bayesian optimization, Gaussian processes, active learning, multi-objective optimization, surrogate models, latent space representations, uncertainty quantification}

\position{Research engineer}{Sberautotech self-driving \href{https://sberautotech.ru/}{info}}{Aug 2020-May 2023}
\project
{Motion trajectory prediction (DL)}
{Led a team (5 ppl) in developing an ML-based agent trajectory prediction project pipeline, overseeing testing and code reviews. Organized weekly research paper review seminars, incorporating ideas from these sessions into final models.}
{Improved metrics for trajectory predictions by 33\% in average, and up to 72\% in special cases. Gathered an in-house dataset with 50k scenes. Developed DL models for predicting agent trajectories by incorporating HD maps and addressing uncertainty through a multi-modal distribution. Deployed model in ROS2 node on self-driving car with average inference time 64ms per frame.}
{python, cpp, pytorch, DL, transformer, CVAE, GAN, CNN, pointnet}

\project
{Model based multi-object tracking / Sensor fusion}
{Research algorithms for multi-object tracking and develop a project pipeline for the task.}
{Established a baseline for model-based tracking by implementing a state-of-the-art algorithm for multi-object tracking. Also developed a simulator for this task with motion and measurement models.}
{Poisson multi Bernoulli mixture, Kalman filter, random finite set, hypothesis tree}

\project
{3D Object detection}
{Research algorithms for 3D object detection and develop first baseline fot this task.}
{Established the initial baseline for 3D object detection using point-cloud data. Trained a model that runs in a ROS2 node on a self-driving car, with an average processing time of 80ms per frame, allowing the self-driving software to understand the presence of objects.}
{python, pytorch, ML, DL, Computer Vision, point cloud, LIDAR, ROS2}

\par \noindent
\position{Software engineer}{Contracted work and other}{Nov 2017 - Aug 2020}
\project
{Multiple object tracking toolbox (> 200 stars) (\href{https://github.com/neer201/Multi-Object-Tracking-for-Automotive-Systems-in-python}{github})}
{}
{Created an open-source Python library for multiple object tracking, implementing the Poisson Multi-Bernoulli Mixture Filter, that is state of the art.}
{}


\project
{End-to-end optical character recognition (OCR) (\href{https://github.com/kharitonov-ivan/end2end_OCR}{github})}
{}
{Worked in a 3-person team, guided by Yandex researchers (Yandex School of Data Analysis project), to enhance text recognition using the EAST model and a text alignment module. We innovated by training detection and recognition simultaneously, improved OCR model efficiency.}
{python, PyTorch, NLP, Computer Vision, OCR}

\project
{Space junk simulator for RL (\textit{project at the Yandex School of Data Analysis} - \href{https://github.com/neer201/space_junk_simulator}{github})}
{}
{Developed a high-performance simulation environment for training reinforcement learning agents in space debris scenarios. Created a parallel-computed physics engine to accurately model orbital dynamics and debris behavior. Implemented reward functions and state representations enabling RL agents to learn optimal satellite control policies for collision avoidance in complex space environments.}
{cpp, python, RL, CUDA, simulation, Runge-Kutta methods, environment design}

\position{Research Engineer}{Central Scientific Research Automotive Institute - \href{https://nami.ru/en/}{info}}{Nov 2015 -- Nov 2017}
\project
{Camera-based object detection  (\href{https://www.engadget.com/2016/08/28/yandex-teams-on-self-driving-shuttle-bus/}{Shuttle project})}
{}
{Developed object detection models for a self-driving shuttle bus. Built dataset, designed metrics, trained, and deployed the model on the shuttle.}
{python, CAFFE, Computer Vision, CNN, Object detection, Image/Video Processing, ROS}

\section*{\sectionformat Education}
\textbf{Bauman Moscow State Technical University} - M.S. in Electrical Engineering \hfill 2008 -- 2014 \\[0.05cm]
\textbf{Yandex School of Data Analysis} - Computer Science \hfill 2017 -- 2019 \\[0.05cm]
\textbf{Autonomous Vehicle Workshop at FSG by Waymo} - \href{https://drive.google.com/file/d/1-WxECccxBrRWIvEt9WQeXKTueiF658r7/view?usp=sharing}{certificate} \hfill 2020 \\[0.05cm]

\section*{\sectionformat Activities}
\begin{tabular}{@{}L!{\VRule}R}
	{\textbf{\textsc{Teaching Assistant}}} Reinforcement Learning course in HSE and YSDA                                                                                                                               & 2019 - present \\
	{\textbf{\textsc{Teaching Assistant}}} Deep Learning in Audio course in HSE                                                                                                                                        & 2023 - 2024    \\
	{\textbf{\textsc{Mentor}}} Bachelor's thesis supervision                                                                                                                                                           & 2022 - 2023    \\
	{\textbf{\textsc{Design Judge}}} Formula Student [Autonomous Driving]     \href{https://www.imeche.org/events/formula-student/team-information/fs-ai}{UK}       \href{https://www.formulastudent.de/fsg/}{Germany} & 2019 - 2023    \\
\end{tabular}

\section*{\sectionformat Publications}
\begin{itemize}[leftmargin=*, itemsep=2pt, parsep=0pt, label={}]
	\item Sorokin D.I., Granovskiy A.A., \textbf{Kharitonov I.}, Stokolesov M., Prokofyev I., Adishchev E., Subbotin G., Nurgaliev M. \textit{Magnetic control of tokamak plasmas through deep reinforcement learning with privileged information}. AI4X 2025 Int. Conference, 2025.
	\item Stokolesov, M. S., Nurgaliev, M. R., \textbf{Kharitonov, I. P.}, Adishchev, E. V., Sorokin, D. I., Clark, R., Orlov, D. M. (2025). Reconstructing the Plasma Boundary with a Reduced Set of Diagnostics. arXiv preprint arXiv:2505.10709.
\end{itemize}

\section*{\sectionformat Skills}
\begin{itemize}[leftmargin=*, itemsep=-2pt, parsep=0pt, label={}]
	\item \textbf{Programming:} Python, C++, MATLAB
	\item \textbf{ML/AI:} PyTorch, ML, DL, NLP, GANs, MCTS, GNN, Reinforcement Learning, Computer Vision
	\item \textbf{MLOps/DevOps:} DVC, Spark, Hydra, Docker, CI/CD (Github Actions, Gitlab-CI), Xpra
	\item \textbf{Robotics:} ROS2, LIDAR, Self-Driving, GNSS, RTK
	\item \textbf{Signal Processing:} Point Clouds, Time Series, Image/Video Processing, Kalman Filtering
	\item \textbf{Control:} Linear/Non-Linear Control, Probabilistic Modelling
\end{itemize}
\end{document}
