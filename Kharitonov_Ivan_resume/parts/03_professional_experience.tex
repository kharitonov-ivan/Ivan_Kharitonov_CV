\section*{Professional Experience}
\begin{tabular}{@{}L!{\VRule}R}
    % Sberautotech
    Aug 2020 -- now                                                                                                       \\ {\bf Sberbank} \\ \boldgrey{Sberautotech \\} &
    {\textsc{Software engineer - perception team}}

    \begin{itemize}
        \item Implemented models for point-cloud based object detection task.
    \end{itemize}
    
    \textsc{Software engineer - prediction team}
    \begin{itemize}
        \item Multi-object tracking -- solve object tracking problem using random finite set statistics.
    \end{itemize}                                                                                                  \\
    % 
    % FSUE NAMI
    Jul 2015 -- Nov 2017                                                                                                       \\ {\bf FSUE NAMI} \\ \boldgrey{Central Scientific Research Automotive Institute -- \\ Information and Intelligent Systems Center} &
    {\textsc{Research engineer at self-driving department (\href{https://www.engadget.com/2016/08/28/yandex-teams-on-self-driving-shuttle-bus/}{Shuttle project})}}
    % 
    \begin{itemize}
        \item Implemented perception models for object detection task -- collecting/generating training data, optimizing the model design, model implementation (Caffe DL framework) and evaluation.
    \end{itemize}
    \textsc{Software developer (Control systems) at transmission control systems department (\href{https://en.wikipedia.org/wiki/Aurus_Senat}{Aurus project})}
    \begin{itemize}
        \item System identification -- created plant models for some vehicle mechanism, such that gearbox clutch hydraulic actuator.
        \item Implemented basic software layer for automotive microcontroller (C, Simulink, Altium Designer) from scratch.
        \item Designed and implemented a controller for hydraulic actuators with further improving quality metrics and decreasing system setting time.
        \item Decreased calibration time by developing automated calibration procedure of control system parameters and tested control algorithms on the testbench.
    \end{itemize}  
\end{tabular}                                                                                                \\
    %
\begin{tabular}{@{}L!{\VRule}R} 
    % FORMULA STUDENT
    Mar 2013 -- Aug 2015                                                                                                       \\ {\bf BMSTU \\ \boldgrey{Bauman Moscow State Technical University}} &
    {\textsc{Hardware and telemetry engineer on \href{https://baumanracing.ru/en/}{an FSAE team}.}}
    Participated in international engineering competition FSAE as a member of the university racing team.
    % 
    Responsibilities: hardware and software development, sponsorship and partnership management.
    % 
    Achievements:
    \begin{itemize}
        \item Released projects: MS thesis -- using RTK navigation for telemetry, F1-like steering wheel with integrated LCD, wireless telemetry module, signals expansion module by reverse-engineering the race ECU CANbus protocol.
        \item Received positive feedback from judges on the design event with good score.
        \item Established sponsorship contracts with several companies. As a result, we were granted new equipment.
    \end{itemize}                                                                                              \\
    % 
    % CRYPTO LLC
    Feb 2012 -- Jul 2013                                                                                                       \\ {\bf Crypto LLC \\ \boldgrey{Systems integrator}} &
    {\textsc{Engineer at System Integration Department.}}
    \begin{itemize}
        \item Adapted the product to the customer by adding fault tolerance setup.
        \item Integrated the monitoring tool (Zabbix) with a data management system.
    \end{itemize}                                                                                                  \\
    % 
    % Dominanta Vimpelcom
    May 2009 -- Feb 2012 {\bf PJSC VimpelCom} &
    {\textsc{Test Engineer.}} The Vimpelcom`s \href{https://www.dvb.org/news/russia-to-launch-dvb-h-services}{pilot project} - TV provider for mobile phones.
    \begin{itemize}
        \item Monitoring of the head and base stations (DVB-H) and 2nd level technical support.
    \end{itemize}
\end{tabular}