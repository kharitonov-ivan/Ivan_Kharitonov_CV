\section*{\sectionformat Professional Experience}
% 
% Sberautotech
\ressubheading{Sberautotech}{(self-driving tech company - \href{https://sberautotech.ru/}{info})}{Moscow}{Research engineer}{Aug 2020 -- Present}

\textbf{Agents trajectory prediction and Scene understanding} \textit{(Prediction)}
\begin{itemize}
    \item Lead a team on ML-based agent trajectory prediction. Developed a project pipeline, testing, code-review.
    \item Implemented physical, seq-to-seq, generative models that incorporate hd map use and deal with uncertainty by considering multi modal distribution.
    \item In-house dataset: Developed a library for use in training agent trajectory prediction models. Gathering data from raw source.
\end{itemize}

\textbf{Multi-object tracking} \textit{(Prediction)}
\begin{itemize}
    \item Analysis of several methods from the literature and finding a suitable solution for, object tracking problem using random finite set statistics. Developed a python library with multi-object tracking algorithms, motion and measurement models, and test scenarios. Consulting a team and sharing concepts of this task.
\end{itemize}

\textbf{3D Object detection} (\textit{Perception})
\begin{itemize}
    \item Implemented ROS2 node for object detection based on point-cloud data stream from LIDARs.
\end{itemize}

\textbf{Other}
\begin{itemize}
    \item Organized weekly research papers reading seminars. Ideas from these seminars used in the production models later.
\end{itemize}

Using: \mymk{numpy}, \mymk{torch}, \mymk{hydra},
\mymk{Kalman filter}, \mymk{GAN}, \mymk{CVAE}, \mymk{ROS2}, \mymk{Docker},
\mymk{Python}, \mymk{gitlab-CI}
\horizontalline
% 
% NAMI
\placeheadding{Central Scientific Research Automotive Institute - FSUE NAMI}{- \href{https://nami.ru/en/}{info}}{Moscow}
\positionheading{Software engineer at self-driving department (\href{https://www.engadget.com/2016/08/28/yandex-teams-on-self-driving-shuttle-bus/}{Shuttle project})}{Jul 2017 -- Nov 2017}
\begin{itemize}
    \item Implemented perception models for object detection task -- collecting/generating training data, optimizing the model design, model implementation (Caffe DL framework) and evaluation.
\end{itemize}
Using: \mymk{Python}, \mymk{CAFFE}

\positionheading{Software engineer at transmission control systems department (\href{https://en.wikipedia.org/wiki/Aurus_Senat}{Aurus project})}{Nov 2015 -- Jul 2017}
\begin{itemize}
    \item System identification -- created plant models for some vehicle mechanism, namely a gearbox clutch hydraulic actuator.
    \item Implemented basic software layer for the automotive microcontroller (C, Simulink, Altium Designer) from scratch.
    \item Designed and implemented a controller for hydraulic actuators with further improving quality metrics and decreasing system setting time.
    \item Speed up calibration by an automated calibration procedure of control system parameters on the testbench.
\end{itemize}
Using: \mymk{MATLAB}, \mymk{Simulink}, \mymk{Vector CANAPE}, \mymk{CAFFE}, \mymk{CAN-BUS}, \mymk{Laplace transform}, \mymk{Model reference control}
\horizontalline
% 
% FSAE
\ressubheading
{Bauman Moscow State Technical University}
{}
{Moscow}
{Engineer at FSAE team (international engineering competition) - \href{https://baumanracing.ru/en/}{info}}
{Mar 2013 -- Aug 2015}
\begin{itemize}
    \item Released projects: MS thesis -- using RTK navigation for telemetry, F1-like steering wheel with integrated LCD, wireless telemetry module, signals expansion module by reverse-engineering the race ECU CANbus protocol.
    \item Received positive feedback from judges at the design event with a good score.
    \item Established sponsorship contracts with several companies. As a result, we were granted new equipment.
\end{itemize}
Using: \mymk{RTKLIB}, \mymk{QT}, \mymk{Python}, \mymk{ARM}, \mymk{Linux}, \mymk{Arduino}, \mymk{Motec}, \mymk{STM32}, \mymk{CAN-BUS}
\horizontalline
% \pagebreak
% 
% CRYPTO LLC
\ressubheading{CRYPTO LLC}{}{Moscow}{Engineer at System Integration Department}{Feb 2012 -- Jul 2013}
\begin{itemize}
    \item Implemented a feature of fault tolerance setup to customer solution.
    \item Integrated a monitoring tool with a data management system.
\end{itemize}
Using: \mymk{Zabbix}, \mymk{Linux}, \mymk{MPEG}, \mymk{Xen}, \mymk{Heartbeat}
\horizontalline
% 
% Dominanta Vimpelcom
\ressubheading
{PJSC VimpelCom}
{providing TV for watching on mobile phones - \href{https://www.dvb.org/news/russia-to-launch-dvb-h-services}{info}}
{Moscow}
{Engineer at infrastructure department}{May 2009 -- Feb 2012}
\begin{itemize}
    \item Setup and calibration of 35 base stations in Moscow.
    \item Monitoring of the head and base stations (DVB-H) and 2nd level technical support.
\end{itemize}
Using: \mymk{Wireshark}, \mymk{DVB}, \mymk{Spectral analysis}, \mymk{MPEG}

