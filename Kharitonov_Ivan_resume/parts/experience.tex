\section*{\sectionformat Professional Experience}
% 
% Sberautotech
\ressubheading{Sberautotech}{(self-driving tech company - \href{https://sberautotech.ru/}{info})}{Moscow}{Research engineer}{Aug 2020 -- Present}
\begin{itemize}
    \item Lead ML-based agent trajectory prediction group.
    \item Implemented physical-based, seq2seq and generative-based models.
    \item In-house prediction dataset: Developed a library for use in training agent trajectory prediction models. Gathering data from raw source (logs for more 1 year period) and preparing dataset.
    \item Multi-object tracking: Analysis of several methods from literature and finding a suitable solution, namely solving object tracking problem using random finite set statistics. Developed a python library with multi-object tracking algorithms and test scenarios. Consulting team and share concepts of this task.
    \item Perception: Implemeneted node for point-cloud based object detection task.
    \item Organized weekly research papers reading seminars. Ideas from these seminars found themselves in the production models.
    \item Manage development: writing tests, CI, provide code review and deployment.
\end{itemize}
Using: \mymk{numpy, scipy, mpl}, \mymk{torch-torchscript}, \mymk{hydra}, \mymk{ROS2}, \mymk{Docker}, \mymk{Python}, \mymk{gitlab-CI}
\horizontalline
% 
% NAMI
\placeheadding{Central Scientific Research Automotive Institute - FSUE NAMI}{- \href{https://nami.ru/en/}{info}}{Moscow}
\positionheading{Software engineer at self-driving department (\href{https://www.engadget.com/2016/08/28/yandex-teams-on-self-driving-shuttle-bus/}{Shuttle project})}{Jul 2017 -- Nov 2017}
\begin{itemize}
    \item Implemented perception models for object detection task -- collecting/generating training data, optimizing the model design, model implementation (Caffe DL framework) and evaluation.
\end{itemize}
Using: \mymk{Python}, \mymk{CAFFE}


\positionheading{Software engineer at transmission control systems department (\href{https://en.wikipedia.org/wiki/Aurus_Senat}{Aurus project})}{Nov 2017 -- Jul 2017}
\begin{itemize}
    \item System identification -- created plant models for some vehicle mechanism, such that gearbox clutch hydraulic actuator.
    \item Implemented basic software layer for the automotive microcontroller (C, Simulink, Altium Designer) from scratch.
    \item Designed and implemented a controller for hydraulic actuators with further improving quality metrics and decreasing system setting time.
    \item Decreased calibration time by developing automated calibration procedure of control system parameters and tested control algorithms on the testbench.
\end{itemize}
Using: \mymk{MATLAB}, \mymk{Simulink}, \mymk{Vector CANAPE}, \mymk{CAFFE}, \mymk{CAN-BUS}, \mymk{Laplace transform}, \mymk{Model reference adaptive control}
\horizontalline
% 
% FSAE
\ressubheading
{Bauman Moscow State Technical University}
{}
{Moscow}
{Engineer at FSAE team (international engineering competition) - \href{https://baumanracing.ru/en/}{info}}
{Mar 2013 -- Aug 2015}
\begin{itemize}
    \item Released projects: MS thesis -- using RTK navigation for telemetry, F1-like steering wheel with integrated LCD, wireless telemetry module, signals expansion module by reverse-engineering the race ECU CANbus protocol.
    \item Received positive feedback from judges on the design event with a good score.
    \item Established sponsorship contracts with several companies. As a result, we were granted new equipment.
\end{itemize}
Using: \mymk{RTKLIB}, \mymk{QT}, \mymk{Python}, \mymk{ARM}, \mymk{Linux}, \mymk{Arduino}, \mymk{Motec}, \mymk{STM32}, \mymk{CAN-BUS}
\pagebreak
\\
% 
% CRYPTO LLC
\ressubheading{CRYPTO LLC}{}{Moscow}{Engineer at System Integration Department}{Feb 2012 -- Jul 2013}
\begin{itemize}
    \item Added feature of fault tolerance setup to customer solution.
    \item Integrated the monitoring tool with a data management system.
\end{itemize}
Using: \mymk{Zabbix}, \mymk{Linux}, \mymk{MPEG}, \mymk{Xen}, \mymk{Heartbeat}
\horizontalline
% 
% Dominanta Vimpelcom
\ressubheading
{PJSC VimpelCom}
{providing TV for watching on mobile phones - \href{https://www.dvb.org/news/russia-to-launch-dvb-h-services}{info}}
{Moscow}
{Engineer at infrastructure department}{May 2009 -- Feb 2012}
\begin{itemize}
    \item Setup and calibration of 35 base stations in Moscow.
    \item Monitoring of the head and base stations (DVB-H) and 2nd level technical support.
\end{itemize}
Using: \mymk{Wireshark}, \mymk{DVB}, \mymk{Spectral anylysis}, \mymk{MPEG}

