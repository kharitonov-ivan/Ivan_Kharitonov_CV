\section*{\sectionformat Experience}
% 
% Sberautotech
\ressubheading{Sber Automotive Technologies}{(self-driving tech company - \href{https://sberautotech.ru/}{info})}{Moscow}{Research engineer}{Aug 2020 -- Present}

Led a team focused on ML-based agent trajectory prediction, by developing a project pipeline and overseeing testing and code review. Additionally, organized weekly seminars to review research papers, with ideas from these sessions being incorporated into the final production models.

\textbf{Motion trajectory prediction} \textit{(Prediction)}
\begin{itemize}
    \item Developed deep learning models for predicting agent trajectories by incorporating HD maps and addressing uncertainty through multi-modal distribution. The models were trained using an in-house dataset that was collected by filtering data from raw logs.
\end{itemize}

\textbf{Model based multi-object tracking} \textit{(Prediction)}
\begin{itemize}
    \item Researched various techniques from literature to identify an appropriate solution for object tracking, leveraging random finite set statistics. Built a Python library that includes multi-object tracking algorithms, motion and measurement models, and test scenarios. Collaborated with a team and shared key concepts related to the task.
\end{itemize}

\textbf{3D Object detection} (\textit{Perception})
\begin{itemize}
    \item Created a ROS2 node that enables real-time object detection using point-cloud data streams sourced from LIDARs.
\end{itemize}
Using: \mymk{python}, \mymk{Kalman filter}, \mymk{GAN}, \mymk{CVAE}, \mymk{ROS2}, \mymk{Docker}, \mymk{gitlab-CI}, \mymk{Spark}, \mymk{numpy}, \mymk{torch}, \mymk{hydra}
\horizontalline
% 
% NAMI
\placeheadding{Central Scientific Research Automotive Institute - FSUE NAMI}{- \href{https://nami.ru/en/}{info}}{Moscow}
\positionheading{Software engineer}{Nov 2015 -- Nov 2017}

\textbf{Camera-based object detection} (\href{https://www.engadget.com/2016/08/28/yandex-teams-on-self-driving-shuttle-bus/}{Shuttle project})
\begin{itemize}
    \item Developed object detection models by gathering and creating training data, refining the model architecture and conducting thorough evaluations.
\end{itemize}

\textbf{Transmission control system} (\href{https://en.wikipedia.org/wiki/Aurus_Senat}{Aurus project})
\begin{itemize}
    \item Performed system identification and modeled various vehicle mechanisms such as gearbox clutch hydraulic actuator. Developed and deployed advanced controller for hydraulic actuators, resulting in 1.2x improvement in quality metrics and up to 40\% reduction in system setting time.
    \item Introduced automated calibration procedures for control system parameters on the testbench, which led to a remarkable improvement in calibration time of up to 70\%.
\end{itemize}
Using: \mymk{python}, \mymk{MATLAB}, \mymk{Simulink},  \mymk{Laplace transform}, \mymk{model reference control}, \mymk{Vector CANAPE}, \mymk{CAN bus}, \mymk{CAFFE}
\horizontalline
% 
% FSAE
\ressubheading
{Bauman Moscow State Technical University}
{}
{Moscow}
{Engineer at FSAE team (international engineering competition) - \href{https://baumanracing.ru/en/}{info}}
{Mar 2013 -- Aug 2015}
\begin{itemize}
    \item Developed a cost-effective localization module utilizing affordable GNSS receivers that leveraged real-time kinematics technology to achieve a high level of accuracy, measuring 10 cm. This solution was particularly notable for its low cost, priced at just 200 USD, making it five times cheaper than comparable devices on the market.
    \item Designed and developed cutting-edge engineering solutions including an F1-inspired steering wheel with an integrated LCD display, a wireless telemetry module, and a signals expansion module developed through reverse-engineering the race ECU CANbus protocol.
    \item Earned high marks and positive feedback from judges at a design event. Additionally, secured sponsorships with multiple companies which enabled us to acquire new equipment to support future projects.
\end{itemize}
Using: \mymk{Linux}, \mymk{python}, \mymk{RTKLIB}, \mymk{QT}, \mymk{ARM}, \mymk{Arduino}, \mymk{Motec}, \mymk{STM32}, \mymk{CAN bus}
\horizontalline
% \pagebreak
% 
% CRYPTO LLC
\ressubheading{CRYPTO LLC}{}{Moscow}{Engineer at System Integration Department}{Feb 2012 -- Jul 2013}
\begin{itemize}
    \item Developed a fault tolerance feature for a customer solution and integrated a monitoring tool with a data management system to improve system reliability and performance.
\end{itemize}
Using: \mymk{Linux}, \mymk{Zabbix}, \mymk{MPEG}, \mymk{Xen}, \mymk{Heartbeat}
\horizontalline
% 
% Dominanta Vimpelcom
\ressubheading
{PJSC VimpelCom}
{(providing TV for watching on mobile phones - \href{https://www.dvb.org/news/russia-to-launch-dvb-h-services}{info})}
{Moscow}
{Engineer at infrastructure department}{May 2009 -- Feb 2012}
\begin{itemize}
    \item Successfully completed the setup and calibration of 35 base stations located in Moscow, including ongoing monitoring of both head and base stations for DVB-H technology, along with providing second-level technical support.
\end{itemize}
Using: \mymk{Wireshark}, \mymk{DVB}, \mymk{Spectral analysis}, \mymk{MPEG}

