\section*{\sectionformat Projects}
\textbf{Object tracking toolbox} (\href{https://github.com/neer201/Multi-Object-Tracking-for-Automotive-Systems-in-python}{github}) \hfill 2021-current
\par
Developed an object tracking library using random finite sets.
Using: \mymk{python}, \mymk{numpy}
\\
% 
\textbf{Race telemetry toolbox} \textit{(contract work)} \hfill 2021
\par
At the annual racecar championship, there is a need to make it possible for the competitors to be on equal terms by equipping them with weight handicap settings, tyres, chassis and powertrain settings.
Developed a service which gathers data from all cars and racetracks, calculates metrics and shows visualization that helps to calculate a required setup for each car.
Using: \mymk{pandas}, \mymk{geopandas}, \mymk{shapely}, \mymk{PyProj}, \mymk{Docker}, \mymk{CI - Github Actions}, \mymk{Balance of Power}
\\
\textbf{End-to-end optical character recognition (OCR)} (\textit{project at the YSDA} - \href{https://github.com/neer201/end2end_OCR}{github})  \hfill 2019
\par
There are usually two stages of text recognition from raw images: detection and recognition. The idea was to train both simultaneously and add an additional module which can help a recognizer to deal with the text alignment. I also reproduced a text detector (EAST)/recognizer model, applied ideas and validated it on the ICDAR dataset.
Using: \mymk{python}, \mymk{torch}
% 
\\
\textbf{Space junk simulator} (\textit{project at the YSDA} - \href{https://github.com/neer201/space_junk_simulator}{github})   \hfill 2019
\par
Developed a simulator of dynamics of space junk objects.
Using: \mymk{CUDA}, \mymk{Runge-Kutta methods}
