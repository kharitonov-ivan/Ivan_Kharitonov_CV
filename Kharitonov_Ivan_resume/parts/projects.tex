\section*{\sectionformat Projects}

\textbf{Multiple object tracking toolbox} (\href{https://github.com/neer201/Multi-Object-Tracking-for-Automotive-Systems-in-python}{github}) \hfill 2021-current

Developed an object tracking library using random finite sets.

Using: \mymk{python}, \mymk{random finite set}, \mymk{numpy}
\\
\textbf{Race telemetry toolbox} \textit{(contract work)} \hfill 2021
\par
Developed a service that collects data from all cars and racetracks at the annual racecar championship to ensure fairness among competitors. The service calculates metrics and provides a visualization that aids in determining the necessary weight handicap, tyre, chassis, and powertrain settings for each car.

Using: \mymk{python}, \mymk{pandas}, \mymk{geopandas}, \mymk{shapely}, \mymk{PyProj}, \mymk{Docker}, \mymk{CI - Github Actions}
\\
\textbf{End-to-end optical character recognition (OCR)} (\textit{project at the Yandex School of Data Analysis} - \href{https://github.com/neer201/end2end_OCR}{github})  \hfill 2019
\par
Developed and tested an approach to improve text recognition from raw images by simultaneously training the detection and recognition stages, and incorporating a text alignment module. This was achieved by adapting an existing text detector (EAST)/recognizer model and evaluating its performance on the ICDAR dataset. The approach was developed within a specific timeframe and is relevant to applications in document processing and image search.

Using: \mymk{python}, \mymk{computer vision}, \mymk{OCR},\mymk{torch}
% 
\\
\textbf{Space junk simulator} (\textit{project at the Yandex School of Data Analysis} - \href{https://github.com/neer201/space_junk_simulator}{github})   \hfill 2019
\par
Created a simulation tool to model the movement and behavior of space debris utilizing the principles of dynamics.

Using: \mymk{cpp}, \mymk{CUDA}, \mymk{Runge-Kutta methods}
