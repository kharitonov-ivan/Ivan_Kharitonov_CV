\section*{Professional Experience}
\vspace{-0.5em}
\begin{tabular}{@{}L!{\VRule}R}
    % Sberautotech
    Aug 2020 -- now                                                                                                                      \\ {\bf Sberbank} \\ \boldgrey{Sberautotech \\} &
    {\textsc{Research software engineer}}
    \begin{itemize}
        \item Prediction: \textbf{agent trajectory prediction} using ML - full cycle development from scratch
              \begin{itemize}
                  \item[--] researching, implement physical-based, seq2seq and generative-based models
                  \item[--] gathering data from raw source in dataset suitable format
                  \item[--] writing tests, CI, provide codereview and setup deployment
              \end{itemize}
        \item Multi-object tracking: Analysis of several methods from literature and find suitable solution - solve object tracking problem using random finite set statistics. Developed python library with multi-object tracking algorithms and test scenarious. Consulting team and share concepts if this task.
        \item Perception: Added models to the pipeline for point-cloud based object detection task.
        \item Other activity: Organize weekly research papers reading seminars. Several ideas from these seminars found themselves in the production models.
    \end{itemize}
    Using: \mymk{numpy, scipy, mpl}, \mymk{torch-torchscript}, \mymk{hydra}, \mymk{ROS2}, \mymk{Docker}, \mymk{Python}, \mymk{gitlab-CI} \\
\end{tabular}
% 
% FSUE NAMI
\begin{tabular}{@{}L!{\VRule}R}
    Jul 2015 -- Nov 2017                                                                      \\ {\bf FSUE NAMI} \\ \boldgrey{Central Scientific Research Automotive Institute -- \\ Information and Intelligent Systems Center} &
    {\textsc{Research engineer at self-driving department (\href{https://www.engadget.com/2016/08/28/yandex-teams-on-self-driving-shuttle-bus/}{Shuttle project})}}
    % 
    \begin{itemize}
        \item Implemented perception models for object detection task -- collecting/generating training data, optimizing the model design, model implementation (Caffe DL framework) and evaluation.
    \end{itemize}
    \textsc{Software developer (Control systems) at transmission control systems department (\href{https://en.wikipedia.org/wiki/Aurus_Senat}{Aurus project})}
    \begin{itemize}
        \item System identification -- created plant models for some vehicle mechanism, such that gearbox clutch hydraulic actuator.
        \item Implemented basic software layer for automotive microcontroller (C, Simulink, Altium Designer) from scratch.
        \item Designed and implemented a controller for hydraulic actuators with further improving quality metrics and decreasing system setting time.
        \item Decreased calibration time by developing automated calibration procedure of control system parameters and tested control algorithms on the testbench.
    \end{itemize}
    Using: \mymk{MATLAB}, \mymk{Simulink}, \mymk{Vector CANAPE}, \mymk{CAFFE}, \mymk{CAN-BUS} \\
\end{tabular}                                                                                                \\
%
\begin{tabular}{@{}L!{\VRule}R}
    % FORMULA STUDENT
    Mar 2013 -- Aug 2015                                                                                                     \\ {\bf BMSTU \\ \boldgrey{Bauman Moscow State Technical University}} &
    {\textsc{Hardware and telemetry engineer on \href{https://baumanracing.ru/en/}{an FSAE team}.}}
    Participated in international engineering competition FSAE.
    % 
    Responsibilities: hardware and software development, sponsorship and partnership management.
    % 
    Achievements:
    \begin{itemize}
        \item Released projects: MS thesis -- using RTK navigation for telemetry, F1-like steering wheel with integrated LCD, wireless telemetry module, signals expansion module by reverse-engineering the race ECU CANbus protocol.
        \item Received positive feedback from judges on the design event with good score. Established sponsorship contracts with several companies. As a result, we were granted new equipment.
    \end{itemize}
    Using: \mymk{RTKLIB}, \mymk{Python}, \mymk{ARM}, \mymk{Linux},\mymk{Arduino}, \mymk{Motec}, \mymk{STM32}, \mymk{CAN-BUS} \\
    % 
    % CRYPTO LLC
    Feb 2012 -- Jul 2013                                                                                                     \\ {\bf Crypto LLC \\ \boldgrey{Systems integrator}} &
    {\textsc{Engineer at System Integration Department.}}
    \begin{itemize}
        \item Adapted the product to the customer by adding fault tolerance setup.
        \item Integrated the monitoring tool with a data management system.
    \end{itemize}
    Using: \mymk{Zabbix}, \mymk{Red hat linux distribution}                                                                  \\
    % 
    % Dominanta Vimpelcom
    May 2009 -- Feb 2012 {\bf PJSC VimpelCom} &
    {\textsc{Test Engineer.}} The Vimpelcom`s \href{https://www.dvb.org/news/russia-to-launch-dvb-h-services}{pilot project} - TV provider for mobile phones.
    \begin{itemize}
        \item Monitoring of the head and base stations (DVB-H) and 2nd level technical support.
    \end{itemize}
    Using: \mymk{Wireshark}, \mymk{DVB}, \mymk{Spectral anylysis tool}
\end{tabular}
